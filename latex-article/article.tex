\documentclass[12pt]{article}
\usepackage{paper}
\usepackage{math}

\title{Example Paper\thanks{\acknlowledgements}}
\author{
  Author One\thanks{School. Email: \href{mailto:email@gmail.com}{email@gmail.com}.} 
  \ and 
  Author Two\thanks{School. Email: \href{mailto:email@gmail.com}{email@gmail.com}.}
}
\date{\textsc{\today}}
\def\acknlowledgements{%
  Acknlowedgements
}

% Conditionally display thoughts (hide by switching to `\boolfalse`)
\booltrue{INCLUDECOMMENTS}
\newcommand{\kyle}[1]{\coauthorComment[Kyle]{#1}}

\begin{document}

% Title Page -------------------------------------------------------------------
\maketitle
\begin{abstract}
  Lorem ipsum dolor sit amet, consectetur adipiscing elit, sed do eiusmod tempor incididunt ut labore et dolore magna aliqua. Ut enim ad minim veniam, quis nostrud exercitation ullamco laboris nisi ut aliquip ex ea commodo consequat. Duis aute irure dolor in reprehenderit in voluptate velit esse cillum dolore eu fugiat nulla pariatur. Excepteur sint occaecat cupidatat non proident, sunt in culpa qui officia deserunt mollit anim id est laborum.

  \par~\par\noindent
  {\color{asher}JEL-Classification:} ...
  \par\noindent
  {\color{asher}Keywords:} ...
  \par\vspace{-2.5mm}
\end{abstract}

\newpage


% Paper ------------------------------------------------------------------------
% ------------------------------------------------------------------------------
\section{Introduction}
% ------------------------------------------------------------------------------

Hi and welcome to my default paper template. I tried to make the theme minimal and beautiful while able to do all the things that I want it to do. These include making it easy to make figures/tables with notes; have powerful math commands; have nice readable typography; make co-authoring in a document easy; have a nice looking bibliography; and make appendices easy.\kyle{I'm including this comment for coauthors. If I switch above to \texttt{togglefalse}, this will dissapear.}  I'll show off these things, but make sure to check the source code alongside to see how simple it is to typset with this. 

See below for \autoref{thm:residue_thm}, the regression specification (\autoref{eq:fe_reg}), \autoref{tab:summ_stat}, \autoref{fig:event_timing}, appendix \autoref{tab:reg1}. Of course, make sure to touch up on your micro theory with \citet{mas1995microeconomic}. I also provide a set of colors: \navy{Navy}, \teal{Teal}, \purple{Purple}, \cranberry{Cranberry}, \orange{Orange}.

% ------------------------------------------------------------------------------
\section{Highlights}
% ------------------------------------------------------------------------------

% ------------------------------------------------------------------------------
\subsection{Math Commands}
% ------------------------------------------------------------------------------

Theorem environments look nice. There are the following environments and their numbering resets automatically for appendices: \texttt{theorem}, \texttt{proposition}, \texttt{assumption}, \texttt{example}, \texttt{lemma}, \texttt{corollary}, \texttt{definition}, \texttt{remark}

\begin{theorem}[Example Theorem]\label{thm:residue_thm}
  This is an example theorem \[
    \hat{\beta}=\frac{\sum_{\ell}e_{\ell}z_{\ell}y_{\ell}^{\perp}}{\sum_{\ell}e_{\ell}z_{\ell}x_{\ell}^{\perp}}
  \]
\end{theorem}

Jibberish math to show off symbols:

\begin{equation}\label{eq:fe_reg}
  y = f(X) + \varepsilon = X \beta + \psi_i + \nu_t + w_{i,t} + \varepsilon_{i,t}
\end{equation}

I've included a file \texttt{math.sty} that has a set of math operators that I find useful.\footnote{Credit to \url{https://pascalmichaillat.org/d3/} for his math commands package which I took almost all of this from.} 

The command \texttt{\textbackslash E[optional]\{optional\}\{optional\}} now lets you specify subscript, the inner term, and a second bracket to do conditional expectation. All three are optional and \textbackslash expec is a alias.
$$\E \quad \E[i] \quad \E{X_i} \quad \E[i]{X_i} \quad \E{X_i}{n} \quad \E[i]{X_i}{n}$$

\texttt{\textbackslash P}, \texttt{\textbackslash prob}, \texttt{\textbackslash cov}, and \texttt{\textbackslash var} work the same way too:
$$\P \quad \P[i] \quad \P{X_i} \quad \P[i]{D_i} \quad \P{\bar{X}_n > \bar{X}}{\mu = \mu_0} \quad \P[\mathcal{F}]{X_i}{D_i = 1}$$
$$\cov{X_n, Y_n} \quad \var[i](\bar{X}_n)$$

\texttt{\textbackslash one} does an indicator. Same as above, but no conditional:
$$\one{X_i > 0} \quad (Y_i, D_i) \indep X_i$$

We have \texttt{\textbackslash aslim}, \texttt{\textbackslash plim}, and \texttt{\textbackslash convd} for convergence symbols. There's also \texttt{\textbackslash iid}. Examples:
$$\bar{x}_n \aslim \mu \quad \bar{x}_n \plim \mu \quad \bar{x}_n \convd N(0, 1)$$
$$X_i \iid N(0, 1)$$

To wrap things in automatically scaling wrappers, can use \texttt{\textbackslash bp} for parenthesis, \texttt{\textbackslash bc} for curly braces, and \texttt{\textbackslash bs} for square brackets:
$$\bp{y_i} \quad \bc{y_i} \quad \bs{y_i}$$

Similar to expectations, I have `upgraded' \texttt{\textbackslash min}, \texttt{\textbackslash inf}, \texttt{\textbackslash liminf}, \texttt{\textbackslash max}, \texttt{\textbackslash sup}, and \texttt{\textbackslash limsup} commands to use the optional `\texttt{[]}' for subscript:
$$\min[i]{x_i} \quad \inf[i]{x_i} \quad \liminf[n \to \infty]{x_i}$$
$$\max[i]{x_i} \quad \sup[i]{x_i} \quad \limsup[n \to \infty]{x_i}$$


% ------------------------------------------------------------------------------
\subsection{Tables}
% ------------------------------------------------------------------------------

For tables, I use the \texttt{tabular} and \texttt{booktabs} packages. For table and figure notes, I use a custom \texttt{\textbackslash note} command. It uses \texttt{\textbackslash parbox} under the hood. You can use it in one of of four ways:
\begin{enumerate}
  \item \texttt{\textbackslash note\{text\}}
  \item \texttt{\textbackslash note[Notes.]\{text\}}
  \item \texttt{\textbackslash note\{0.6\textbackslash textwidth\}\{text\}}
  \item \texttt{\textbackslash note[Notes.]\{0.6\textwidth\}\{text\}}
\end{enumerate}

In addition, I use the \texttt{adjustbox} package for resizing figures/tables. It automatically scales the figure/table proportionally, so things look right. For example, here's a table that's too wide. I use the \texttt{adjustbox} package to fix it.

\begin{table}[ht]
  \caption{Table Too Wide (adjustbox)}\label{tab:summ_stat}
  \centering

  \begin{adjustbox}{width = \textwidth, center}
    \begin{tabular}{ 
      @{} @{\extracolsep{5pt}} l*{8}{r} @{}
    }
      % Head
      \toprule
      & & & \multicolumn{2}{c}{Market Access} & \multicolumn{2}{c}{Urban Weekly Wage} & \multicolumn{2}{c}{Nonurban Weekly Wage} \\
      \cmidrule{4-5} \cmidrule{6-7} \cmidrule{8-9}
      Year & \multicolumn{1}{c}{N} & \% Urban & Mean & SD & Mean & SD & Mean & SD \\
      \hline

      % Body
      1940 & 16,875,829 & 0.66 & 10,708.21 & 14,819.55 & 33.22 & 19.66 & 25.23 & 16.56\\
1950 & 67,790 & 0.69 & 23,166.06 & 26,600.85 & 70.05 & 32.70 & 58.27 & 29.24\\
1960 & 1,338,491 & 0.66 & 40,328.17 & 45,385.47 & 124.11 & 77.55 & 99.37 & 59.01\\
1970 & 277,951 & 0.75 & 65,373.07 & 68,048.88 & 205.42 & 137.80 & 165.61 & 104.49\\
1980 & 1,907,836 & 0.73 & 73,223.43 & 74,906.62 & 408.83 & 284.44 & 340.82 & 220.68\\
1990 & 2,257,874 & 0.68 & 224,312.50 & 168,933.04 & 711.77 & 574.58 & 550.26 & 392.90\\
2000 & 2,581,741 & 0.73 & 288,195.77 & 199,100.92 & 1,010.36 & 1,022.77 & 754.10 & 649.57\\
2010 & 530,359 & 0.76 & 273,754.31 & 195,220.04 & 1,306.70 & 1,281.34 & 972.40 & 817.94\\


      \bottomrule
    \end{tabular}
  \end{adjustbox}

  \note{Weekly wage is reported in 2015 \$.}
\end{table}

Figures also use the \texttt{\textbackslash note} and the \texttt{adjustbox} package. Here's an example figure:

\begin{figure}[b!]
  \caption{Event-timing}\label{fig:event_timing}
  \begin{adjustbox}{width = 0.6\textwidth, center}
    \begin{tikzpicture}
      \draw (0,0) -- (11,0);
      \foreach \x in {0.8,4,5.5,7,10.2}
      \draw(\x cm,3pt) -- (\x cm, -3pt);
      \draw (0.8,0) node[below=3pt] {$T_0$};
      \draw (4,0) node[below=3pt] {$T_1$};
      \draw (5.5,0) node[below=3pt] {$0$};
      \draw (7,0) node[below=3pt] {$T_2$};
      \draw (10.2,0) node[below=3pt] {$T_3$};
      \draw (2.35,0) node[above=6pt, align=center]
      {\begin{tabular}{c}Estimation \\Window\end{tabular}};
    \end{tikzpicture}
  \end{adjustbox}
  
  \begin{center}
    \note{0.6\textwidth}{This is an example figure in the paper}
  \end{center}
\end{figure}

% ------------------------------------------------------------------------------
\newpage
\bibliography{references.bib}
% ------------------------------------------------------------------------------

% ------------------------------------------------------------------------------
\newpage
\appendix
% ------------------------------------------------------------------------------

% ------------------------------------------------------------------------------
\section{Additional Results}
% ------------------------------------------------------------------------------

The appendix will automatically start numbering tables, figures, and theorem-like environments using the appendix section \textbackslash Alph (e.g. \autoref{tab:reg1}).

% ------------------------------------------------------------------------------
\subsection{Regression Results}
% ------------------------------------------------------------------------------

In this table example, the table is narrower than textwidth, so I adjust the \texttt{\textbackslash note} width. 

\begin{table}[h!]
  \caption{Regression Results}
  \label{tab:reg1}
  \begin{center}
    \begin{tabular}{@{} lrr @{}}
      % Head
      \toprule
      & \multicolumn{2}{c}{\textit{Dependent variable:} \ Overall Rating} \\
      \cmidrule{2-3}
                                & (1)                   & (2)                   \\
      \hline

      % Body
      Handling of Complaints   & 0.692$^{***}$ (0.149) & 0.682$^{***}$ (0.129) \\
      No Special Privileges    & $-$0.104 (0.135)      & $-$0.103 (0.129)      \\
      Opportunity to Learn     & 0.249 (0.160)         & 0.238$^{*}$ (0.139)   \\
      Performance-Based Raises & $-$0.033 (0.202)      &                       \\
      Too Critical             & 0.015 (0.147)         &                       \\
      Advancement              & 11.011 (11.704)       & 11.258 (7.318)        \\
      \hline
      Observations             & 30                    & 30                    \\
      Adjusted R$^{2}$         & 0.656                 & 0.682                 \\

      \bottomrule
    \end{tabular}

    \note{0.68\textwidth}{Using R base dataframe attitude. dolor in reprehenderit in voluptate velit esse cillum dolore eu fugiat nulla pariatur. Excepteur sint occaecat cupidatat non proident, sunt in culpa qui officia deserunt mollit anim id est laborum.}
    \note[]{0.68\textwidth}{$^{*}$ $p < 0.1$; $^{**}$ $p < 0.05$; $^{***}$ $p < 0.01$.}
  \end{center}
\end{table}


\end{document}
