
\usepackage[english]{babel}
\usepackage{a4wide}
\usepackage{csquotes}
\usepackage{float, afterpage, rotating, graphicx}
\usepackage{epstopdf}
\usepackage{longtable, booktabs, tabularx}
\usepackage{fancyvrb, moreverb, relsize}
\usepackage{eurosym, calc, chngcntr}
\usepackage{mathtools, amssymb, amsfonts, amsthm, bm,amsmath}
\usepackage[toc,page]{appendix}
\usepackage{mdwlist}
\usepackage{xfrac}
\usepackage{setspace}
\usepackage{xcolor}
\usepackage{enumitem}
\usepackage{ragged2e} % to reverse centering by \justify
\usepackage[margin=1.1in,footskip=0.7in]{geometry}
\usepackage[ruled]{algorithm2e}
\usepackage{tikz}

% to merge rows in the table into one. USAGE:
% age, industry,     &   \multirow{ 2}{*}{Yes}   & \multirow{ 2}{*}{Yes} \
% year fixed effects \
\usepackage{multirow} 



% \usepackage{endfloat} % Enable to move tables / figures to the end. Useful for some submissions.

\usepackage{subfig}

\usepackage[
sortcites=true,
sorting=ynt,
natbib=true,
bibencoding=inputenc,
bibstyle=authoryear,
citestyle=authoryear-comp,
% style=chicago-authordate,
maxcitenames=2,
maxbibnames=10,
useprefix=false,
backend=biber,
doi=false, isbn=false, url=false, eprint=false,
uniquelist=false
]{biblatex}

% \usepackage[authordate,backend=biber]{biblatex-chicago}

\AtBeginDocument{\toggletrue{blx@useprefix}}
\AtBeginBibliography{\togglefalse{blx@useprefix}}
\setlength{\bibitemsep}{1.5ex}

\definecolor{linkCol}{HTML}{454b4d}

\usepackage[unicode=true,hidelinks]{hyperref}
\hypersetup{
colorlinks=false,
linkcolor=black,
anchorcolor=black,
citecolor=black,
filecolor=black,
menucolor=black,
runcolor=black,
urlcolor=black
}

%%%%%%%%%%%% FONT %%%%%%%%%%%%
\usepackage[utf8]{inputenc} % not needed when xelatex
\usepackage[T1]{fontenc}  % not needed when xelatex
%\usepackage[sc]{mathpazo} %  not needed when xelatex
%\usepackage{newpxmath} % not needed when xelatex
%\usepackage[scaled=.98,sups,osf]{XCharter}% osf in text, lining figures in math
%\usepackage[scaled=1.04,varqu,varl]{inconsolata}% inconsolata typewriter
%\usepackage[type1]{cabin}% sans serif
%\usepackage[uprightscript,charter,vvarbb,scaled=1.05]{newtxmath}
\usepackage{lato}
\linespread{1.5}

% this below is only for xelatex
%\usepackage{microtype} % clearer font
%\usepackage{unicode-math}
%\setmathfont{Asana Math}
%\setmainfont{Palatino Linotype}

%\usepackage{fontspec}
%\usepackage{mathspec}
%\setmainfont[
%             UprightFont={*-Regular},
%             BoldFont={*-SemiBold},
%             ItalicFont={*-RegularItalic},
%             BoldItalicFont={*-SemiBoldItalic},
%             Scale=1.0,
%            Numbers=Lining
%            ]{Metropolis}
% \setmonofont[Color={010101},Scale=0.88]{JetBrainsMono}

% \setmathfont(Digits,Latin,Greek)[Scale=MatchLowercase]{STIXTwoMath-Regular}
% \setmathfont(Digits,Latin,Greek)[Scale=MatchUppercase,Numbers={Lining,Proportional}]{latinmodern-math.otf}
% \setmathfont(Digits,Latin,Greek)[Scale=MatchUppercase,Numbers={Lining,Proportional}]{Metropolis-LightItalic}

% this is for pdftex
% \usepackage{fourier}

\definecolor{titlecol}{HTML}{374C8C}
\definecolor{figcol}{HTML}{3067a6}
\definecolor{grey}{HTML}{393b39}

\usepackage{sectsty} % Allows customization of titles
\sectionfont{\Large \color{titlecol}} % Color section titles
\subsectionfont{\large \color{titlecol}} % Color section titles
\subsubsectionfont{\normalsize \color{titlecol}} % Color section titles


% set color of figure titles
\usepackage[labelfont={color=figcol,bf}]{caption}


% define figure notes
\newcommand\fnote[1]{\captionsetup{font={footnotesize}}\caption*{\textcolor{grey}{\emph{Notes}: #1}}}

\newcommand{\linkText}[1]{{\textcolor{linkCol}{#1}}}
\widowpenalty=10000
\clubpenalty=10000

\setlength{\parskip}{0ex}
% \setlength{\parindent}{0ex}

\theoremstyle{plain}

\newtheorem{assumption}{Assumption}
\newtheorem{theorem}{Theorem}
\newtheorem{proposition}[theorem]{Proposition}
\newtheorem{corollary}[theorem]{Corollary}
\newtheorem{lemma}[theorem]{Lemma}
\newtheorem{remark}[theorem]{Remark}
\theoremstyle{definition}
\newtheorem{definition}{Definition}
% \theoremstyle{remark}

%declaration of new variables
\newcommand{\ud}{\mathrm{d}}
\newcommand{\pd}[2]{\frac{\partial #1}{\partial #2}}
\newcommand{\pdinline}[2]{ \partial #1/\partial #2}
\newcommand{\td}[2]{\frac{\ud #1}{\ud #2}}

\renewcommand{\iff}{\ensuremath{\Leftrightarrow}}
\newcommand{\tends}[1][{}]{\overset{#1}{\to}}
\newcommand{\lub}{\ensuremath{\,\vee\,}}
\newcommand{\oraz}{\ensuremath{\,\wedge\,}}
\newcommand{\then}{\ensuremath{\,\Rightarrow \,}}
\newcommand{\neht}{\ensuremath{\Leftarrow}}

\newcommand{\tinf}{_{t=0}^{\infty}}

\newcommand{\R}{\mathbb{R}} %Real numbers
\newcommand{\Se}{\mathcal{S}} % Borel set
\newcommand{\margc}{{MC}} % marginal cost


\newcommand{\opts}{\bar{k}(z)} % optimal size

\newcommand{\bc}[1]{\left\lbrace #1 \right\rbrace} 
\newcommand{\bp}[1]{\left( #1 \right)} 
\newcommand{\bs}[1]{\left[ #1 \right]} 
\newcommand{\abs}[1]{\left\lvert #1 \right\rvert}
\newcommand{\norm}[1]{\left\lVert #1 \right\rVert}
\newcommand{\of}[1]{{\left( #1 \right)}}
\def\detla{\delta}
\def\lamdba{\lambda}

%new mathematical operators
\DeclareMathOperator{\Lagr}{\mathcal{L}}
\DeclareMathOperator{\Ham}{\mathcal{H}}
\DeclareMathOperator*{\argmin}{\operatorname{argmin}}
\DeclareMathOperator*{\argmax}{\operatorname{argmax}}
\DeclareMathOperator{\E}{\mathbb{E}} % expected value
\DeclareMathOperator{\prob}{\mathbb{{P}}}
\DeclareMathOperator{\Cov}{\mathrm{Cov}}
\DeclareMathOperator{\Var}{\mathrm{Var}}
\DeclarePairedDelimiter{\ceil}{\lceil}{\rceil}
\DeclarePairedDelimiter{\floor}{\lfloor}{\rfloor}

%new environments
\newcommand{\myvector}[3]{\begin{bmatrix} {#1} \ {#2} \ \vdots \ {#3} \end{bmatrix}}

%visual customization
\renewcommand{\labelitemi}{$\bullet$}
\renewcommand{\textapprox}{\raisebox{0.5ex}{\texttildelow}}

\newcommand{\fignotes}[1]{ {\footnotesize   Note: #1 } }
\newcommand{\figtitle}[1]{ { \footnotesize \textbf{#1}} }

\renewcommand\appendixpagename{\color{titlecol}{Appendices}}


\def\a{\alpha}
\def\b{\beta}
\def\c{\chi}
\def\d{\delta}
\def\D{\Delta}
\def\uD{\upDelta}
\def\e{\epsilon}
\def\f{\phi}
\def\vf{\varphi}
\def\F{\Phi}
\def\g{\gamma}
\def\G{\Gamma}
\def\h{\eta}
\def\k{\kappa}
\def\l{\lambda}
\def\L{\Lambda}
\def\m{\mu}
\def\n{\nu}
\def\o{\omega}
\def\O{\Omega}
\def\vp{\varpi}
\def\p{\psi}
\def\r{\rho}
\def\s{\sigma}
\def\vs{\varsigma}
\def\S{\Sigma}
\def\t{\theta}
\def\T{\Theta}
\def\vt{\vartheta}
\def\U{\Upsilon}
\def\x{\xi}
\def\X{\Xi}
\def\z{\zeta}
