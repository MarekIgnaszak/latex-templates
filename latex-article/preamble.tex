% Margins ----------------------------------------------------------------------
\usepackage[margin=1.25in]{geometry}

% Line Spacing -----------------------------------------------------------------
\renewcommand{\baselinestretch}{1.5}
\usepackage{setspace}

% Font -------------------------------------------------------------------------
\usepackage[T1]{fontenc}
\usepackage{inputenc}
\usepackage[default]{lato} % Lato as text font
% \usepackage[utopia, varg]{newtxmath}
% \renewcommand{\rmdefault}{futs} % Utopia as text font 

% Small adjustments to text kerning
\usepackage{microtype}

% Remove annoying over-full box warnings
\vfuzz2pt 
\hfuzz2pt

% Color Palette ----------------------------------------------------------------
\usepackage{xcolor}

% High contrast colors
\definecolor{purple}{HTML}{5601A4}
\definecolor{cranberry}{HTML}{E64173}
\definecolor{orange}{HTML}{D65616}
\definecolor{navy}{HTML}{006896}
\definecolor{teal}{HTML}{1A505A}
\newcommand\purple[1]{{\color{purple}#1}}
\newcommand\cranberry[1]{{\color{cranberry}#1}}
\newcommand\orange[1]{{\color{orange}#1}}
\newcommand\navy[1]{{\color{navy}#1}}
\newcommand\teal[1]{{\color{teal}#1}}

% Kyle's colors
\definecolor{ruby}{HTML}{9a2515}
\definecolor{alice}{HTML}{107895}
\definecolor{daisy}{HTML}{EBC944}
\definecolor{coral}{HTML}{F26D21}
\definecolor{kelly}{HTML}{829356}
\definecolor{jet}{HTML}{131516}
\definecolor{asher}{HTML}{555F61}
\definecolor{slate}{HTML}{314F4F}
\newcommand\kelly[1]{{\color{kelly}#1}}
\newcommand\ruby[1]{{\color{ruby}#1}}
\newcommand\alice[1]{{\color{alice}#1}}
\newcommand\daisy[1]{{\color{daisy}#1}}
\newcommand\coral[1]{{\color{coral}#1}}
\newcommand\slate[1]{{\color{slate}#1}}
\newcommand\jet[1]{{\color{jet}#1}}
\newcommand\asher[1]{{\color{asher}#1}}

% Hyperlinks -------------------------------------------------------------------
\usepackage{hyperref}
\hypersetup{
  colorlinks = true,
  allcolors = purple
}

% Citations --------------------------------------------------------------------
% note, natbib provides better hyperlinking
\usepackage{natbib}
\bibliographystyle{econ-aea}

% Conditionals -----------------------------------------------------------------
% https://tex.stackexchange.com/questions/33576/conditional-typesetting-build
\usepackage{etoolbox}

% Titlepage --------------------------------------------------------------------
% \maketitle
\usepackage{titling}

% title
\pretitle{\begin{spacing}{1}\begin{flushleft}\huge}
\posttitle{\end{flushleft}\end{spacing}\vspace{-5mm}}

% author, note don't use \and 
\preauthor{\begin{flushleft}\large}
\postauthor{\end{flushleft}\vspace{-7.5mm}}

% date \scshape
\predate{\begin{flushleft}\color{asher}}
\postdate{\end{flushleft}\vspace{-5mm}}

% Abstract
\renewenvironment{abstract}
 {\noindent\rule{\linewidth}{.5pt}\noindent}
 {\noindent\rule{\linewidth}{.5pt}}

% alternative abstract
% \renewenvironment{abstract}
% {
%   \centerline {\large \bfseries \scshape  Abstract}
%   \begin{quote}
% }
% {\end{quote}}


% Section and Subsection Styling -----------------------------------------------
\usepackage[explicit]{titlesec}

\titleformat{\section}
  {\Large \bf}
  {\thesection \,---}
  {0.25em}
  {#1}
  
\titleformat{\subsection}
  {\fontsize{11}{10}\it}
  {\thesubsection.}
  {1em}
  {#1}


% Footnote ---------------------------------------------------------------------
% Spacing between footnotes on same page
\addtolength{\footnotesep}{1mm}

% Space after footnote number
\let\oldfootnote\footnote
\renewcommand\footnote[1]{\oldfootnote{\ #1}}

% No footnote line
\renewcommand\footnoterule{}

% No supsercript in footer
\makeatletter
\renewcommand\@makefntext[1]{%
    \parindent 1em \noindent
    \hb@xt@1.8em{\hss\normalfont\@thefnmark.\hfill}#1
  }
\makeatother

% Enumerate/Itemize ------------------------------------------------------------
\usepackage{enumitem}
\setitemize{labelindent=0.5em,labelsep=0.25cm,leftmargin=*}
\setenumerate{labelindent=0.5em,labelsep=0.25cm,leftmargin=*}

% tikz support -----------------------------------------------------------------
\usepackage{tikz}
\usepackage{pgfplots}
\pgfplotsset{compat=1.7}

% Table and Figure labelling ---------------------------------------------------
\usepackage{caption}

\DeclareCaptionLabelSeparator{threedash}{\,---\,}
\DeclareCaptionFont{jet}{\color{jet}}
\captionsetup*[table]{format=plain, labelsep=threedash, font={bf}}
\captionsetup*[figure]{format=plain, labelsep=threedash, font={bf}}

% Alternative: Left align captions
% \captionsetup[table]{labelfont=it, textfont={bf}, labelsep=newline, justification=raggedright, singlelinecheck=off}
% \captionsetup[figure]{labelfont=it, textfont={bf}, labelsep=newline, justification=raggedright, singlelinecheck=off}

% multifigure with \caption
% \begin{subfigure}\caption{} \end{subfigure}
\usepackage{subcaption}
\captionsetup*[subfigure]{format=plain, font={jet, footnotesize, bf}}


% Tables -----------------------------------------------------------------------
% Fix \input with tables
% \input fails when \\ is at end of external .tex file
\makeatletter
\let\input\@@input
\makeatother

% Make tables/figures wider than \textwidth using:
% \begin{adjustbox}{width = 1.2\textwidth, center}
% \end{adjustbox}
\usepackage{adjustbox}

% Slighty more spacing between rows
\usepackage{array}
\renewcommand\arraystretch{1.2}

% Table with easy to use footnotes
% \begin{threeparttable}
%    \begin{tabular} ... \end{tabular}
%    \begin{tablenotes}
%        \item \textit{Notes.}
%    \end{tablenotes}  
% \end{threeparttable}
\usepackage[flushleft]{threeparttable}
\setlength\labelsep{0pt}

% \toprule, \cmidrule, \bottomrule
\usepackage{booktabs}

% If tables are too narrow, fill columns using:
% \begin{tabularx}{\linewidth}{cols}
% col-types: X - center, L - left, R -right
% If you want relative scale for columns: 
% >{\hsize=.8\hsize}X/L/R
\usepackage{tabularx}
\newcolumntype{L}{>{\raggedright\arraybackslash}X}
\newcolumntype{R}{>{\raggedleft\arraybackslash}X}
\newcolumntype{C}{>{\centering\arraybackslash}X}

% Landscape table 
% \begin{landscape} \pagestyle{lscaped} table... \end{landscsape}
% \usepackage{pdflscape} - rotates page left-side up in pdf
% \usepackage{lscape} - does not rotate page, only figure/table

\usepackage{pdflscape}

% For landscape, fix page number location
\usepackage{fancyhdr}
\fancypagestyle{lscaped}{%
    \fancyhf{}
    \renewcommand{\headrulewidth}{0pt}
    \textnormal
    \fancyfoot{%
        \tikz[remember picture,overlay]
        \node[outer sep=2.5cm,above,rotate=90] at (current page.east) {\thepage};
    }
}

% Math stuff -------------------------------------------------------------------
\usepackage{amsmath}
\usepackage{amsfonts}
\usepackage{amsthm}
\usepackage{graphicx}
\usepackage{bm}

% Define Theorems --------------------------------------------------------------
% Put proper spacing after Theorem #. 
\newtheoremstyle{spacing}
{}%          Space above, empty = `usual value'
{}%          Space below
{}%  Body font
{}%          Indent amount (empty = no indent, \parindent = para indent)
{\bfseries}% Thm head font
{.}%         Punctuation after thm head
{2.5mm}%  Space after thm head: \newline = linebreak
{}%          Thm head spec

% note, theorem is the name that goes in \begin{} and Theorem is the name displayed as Theorem 1
\theoremstyle{spacing}
\newtheorem{theorem}{Theorem}
\newtheorem{proposition}{Proposition}
\newtheorem{assumption}{Assumption}
\newtheorem{example}{Example}


% Custom Math Definitions ------------------------------------------------------
\newcommand{\expec}[1]{\mathbb{E}\left[#1\right]}%
\newcommand{\condexpec}[2]{\mathbb{E}\left[#1 \ \vert \ #2\right]}%
\newcommand{\prob}[1]{\text{Pr}\left[#1\right]}%
\newcommand{\condprob}[2]{\text{Pr}\left[#1 \ \vert \ #2\right]}%
\newcommand{\var}[1]{\mathrm{Var}\left[#1\right]}%
\newcommand{\cov}[1]{\mathrm{Cov}\left[#1\right]}%
\newcommand{\one}{\mathbf{1}}
